%% Based on a TeXnicCenter-Template by Tino Weinkauf.
%%%%%%%%%%%%%%%%%%%%%%%%%%%%%%%%%%%%%%%%%%%%%%%%%%%%%%%%%%%%%

%%%%%%%%%%%%%%%%%%%%%%%%%%%%%%%%%%%%%%%%%%%%%%%%%%%%%%%%%%%%%
%% HEADER
%%%%%%%%%%%%%%%%%%%%%%%%%%%%%%%%%%%%%%%%%%%%%%%%%%%%%%%%%%%%%
\documentclass[letterpaper,twoside,11pt]{article}
% Alternative Options:
%	Paper Size: a4paper / a5paper / b5paper / letterpaper / legalpaper / executivepaper
% Duplex: oneside / twoside
% Base Font Size: 10pt / 11pt / 12pt


%% Language %%%%%%%%%%%%%%%%%%%%%%%%%%%%%%%%%%%%%%%%%%%%%%%%%
\usepackage[USenglish]{babel} %francais, polish, spanish, ...
\usepackage[T1]{fontenc}
\usepackage[ansinew]{inputenc}

\usepackage{lmodern} %Type1-font for non-english texts and characters

\usepackage{hyperref}
%% Packages for Graphics & Figures %%%%%%%%%%%%%%%%%%%%%%%%%%
\usepackage{graphicx} %%For loading graphic files
%\usepackage{subfig} %%Subfigures inside a figure
%\usepackage{pst-all} %%PSTricks - not useable with pdfLaTeX

%% Please note:
%% Images can be included using \includegraphics{Dateiname}
%% resp. using the dialog in the Insert menu.
%% 
%% The mode "LaTeX => PDF" allows the following formats:
%%   .jpg  .png  .pdf  .mps
%% 
%% The modes "LaTeX => DVI", "LaTeX => PS" und "LaTeX => PS => PDF"
%% allow the following formats:
%%   .eps  .ps  .bmp  .pict  .pntg

%% Math Packages %%%%%%%%%%%%%%%%%%%%%%%%%%%%%%%%%%%%%%%%%%%%
\usepackage{amsmath}
\usepackage{amsthm}
\usepackage{amsfonts}


%%%%%%%%%%%%%%%%%%%%%%%%%%%%%%%%%%%%%%%%%%%%%%%%%%%%%%%%%%%%%
%% DOCUMENT
%%%%%%%%%%%%%%%%%%%%%%%%%%%%%%%%%%%%%%%%%%%%%%%%%%%%%%%%%%%%%
\begin{document}

%\pagestyle{empty} %No headings for the first pages.


%% Title Page %%%%%%%%%%%%%%%%%%%%%%%%%%%%%%%%%%%%%%%%%%%%%%%
%% ==> Write your text here or include other files.

%% The simple version:
\title{Socialwiki Documentation}
\author{Jeremy Dunsmore}
\maketitle

%% Inhaltsverzeichnis %%%%%%%%%%%%%%%%%%%%%%%%%%%%%%%%%%%%%%%
%\tableofcontents %Table of contents
%\cleardoublepage %The first chapter should start on an odd page.

%\pagestyle{plain} %Now display headings: headings / fancy / ...

\section{Peer}
 A peer has 6 attributes:
	\begin{itemize}
		\item id the userid of the user the peer represents
		\item	trust a number based on following relations 
		\item	likesim a percentage of how similar the peers likes are to the user's
		\item	followsim a percentage of how similar the peers follows are to the user's
		\item popularity percent popularity  
		\item weight an associative array of the weights of each of the 4 scores
	\end{itemize}

\section{Node}
A node is used to describe a page in a tree.
\subsection{Attributes}
\begin{itemize}
	\item id: the nodes id
	\item content: the title and author of the page
	\item hidden: used in JavaScript to determine if node is to be hidden
	\item column: used in JavaScript to determine which column the node is placed in
	\item children: an array of the ids of the nodes children
	\item parent: the id of the nodes parent -1 if it has no parent
	\item level: the depth of a node. ex root is 1 root's children 2 etc...
	\item added used in JavaScript to determine if the node has been added to the tree
	\item priority: used to determine the order the socialwiki\_tree nodes
\end{itemize}
\subsection{Methods}
	\begin{itemize}
		\item \_\_construct: takes a page sets column, added, hidden and content then sets priority to the page's votes
		\item set\_content: gets the page's author, title and creates the HTML for a link to the page.
		\item add\_child: takes a child id and appends it to the children array.
		\item display: display's the content in an HTML box.
	\end{itemize}



\section{Socialwiki\_tree}
The socialwiki\_tree is used to show the relations between pages on the history and search pages.
The socialwiki\_tree has 1 attribute an associative array of socialwiki\_nodes.
	\subsection{methods}
		\begin{itemize}
			\item build\_tree: given an array of pages builds a tree by adding all the pages as nodes and calling add\_children to add the child arrays to all the nodes.
			\item add\_node: given a page adds the page as a node to the nodes array with the id being l+pageid
			\item add\_children: goes through the nodes array and builds the children arrays for each node
			\item sort: organizes the array by ordering leaves in order of priority then organizes the nodes in to a tree placing the root node first followed by its children
			\item add\_levels: goes through the array after it has been sorted and adds a level to each node root gets 1 root's child 2 etc.
			\item repos\_children: this function gets called when sorting the array and when a parent is moved it places all of the children directly after the parent in the array.
			\item add\_parent: recursively adds nodes to the array positioning them based on child location and priority.
			\item find\_ leaves: returns an array of nodes with no children
			\item find\_index: finds sibling locations and determines, based on priority, where the index of the array the node should be placed at.
			\item display: calls sort then display the tree
		\end{itemize}

\section{Peerscore Algorithm}
	The like and follow similarity for a peer are calculated using: 
		\[
		\frac{|likes (p1)\cap likes(user)|}{|likes(p1)\cup likes (user)|}
	\]
	The trust of a peer is 1 over the following distance of a peer.
	\newline
	\\Peer popularity is calculated by: 
		\[
		\frac{num\ following}{num\ peers}
	\]
	The peerscore is then calculated using:
	
		\[
		\sum score_i\times weight_i
	\]
	
\section{Pagescore}
	The page score is calculated as $(\sum peerscore)+\frac{time\ created}{current\ time} $
	
\section{New Database Tables}
	\subsection{socialwiki\_follows}
	The follows table has 4 columns:
	\begin{itemize}
		\item id
		\item userfromid: the userid of the user doing the following
		\item usertoid: the userid of the user being followed
		\item subwikiid: the subwikiid for the activity
	\end{itemize}
	\subsection{socialwiki\_likes}
	The likes table has 4 columns:
	\begin{itemize}
		\item id
		\item userid: the userid of the user doing the liking
		\item pageid: the pageid of the page being liked
		\item subwikiid: the subwikiid for the activity 
	\end{itemize}
	
	\subsection{socialwiki}
	A style column was added to the socialwiki table to track which style the module is to use.
	
\section{Files and Folders}
\subsection{Files in socialwiki root folder}
\begin{itemize}
	\item \textbf{admin.php}
	This file is used for displaying the administration tab.
	\item \textbf{ajax.php}
	This file is used to process ajax calls from search.js
	\item \textbf{classic\_style.css and modern\_style.css}
	These files are used to format the output of the two different styles classic and modern.
	\item \textbf{styles.css}
	This file contains the css that is used in all styles
	\item \textbf{comments.php}
	This file is used for the comments page.
	\item \textbf{create.php}
	This file is used to create a new wiki page
	\item \textbf{diff.php}
	This file is  used to compare two different pages from the same tree.
	\item \textbf{edit.php}
	This file is used to edit a page
	\item \textbf{follow.php}
	This file is used to add or remove a follow.
	\item \textbf{help.php}
	This file is used to create the help tab
	\item \textbf{history.js}
	This file contains all of the JavaScript used for the history page.
	\item \textbf{history.php}
	This file is used to create the history page.
	\item \textbf{home.php}
	This file is used to create the home page.
	\item \textbf{lib.php}
	This file contains functions that are called by the moodle system
	\item \textbf{locallib.php}
	This is one of the more important files, it contains all the function definitions that are used within the activity.
	\item \textbf{manage.php}
	This file is used to create the manage page.
	\item \textbf{pagelib.php}
	This is one of the most important files in this module as it contains all of the class definitions for the pages used in the other php files.
	\item \textbf{renderer.php}	
	This file is used to create the HTML for certain functions.
	\item \textbf{search.php}
	This file is used for the search page.
	\item \textbf{search.js}
	This file contains all of the JavaScript for the search page. Creates both the tree for tree view and the weight sliders for tree view and list view.
	\item \textbf{toolbar.js}
	This file contains all the JavaScript for the toolbar used in the modern style.
	\item \textbf{version.php}
	This file is used to track the current version of socialwiki as well as the moodle version required to install the plug-in. The current moodle version required is 2.5.
	\item \textbf{view.php}
	This page is used to view a wiki page and is the first file called when the user enters the activity. However the file just redirects to the home page when this happens.
	\item \textbf{viewuserpages.php}
	This page is used to create the user profile page that gets called when a user clicks on the author of a page.
\end{itemize}	
	\subsection{backup folder}
	this folder contains all the php files required to backup a socialwiki activity
	currently only backup for moodle 2 is supported.
	\subsection{db folder}
	This folder has all the files that describe the database.
\begin{itemize}	
\item \textbf{access.php}
	This file contains all of the permissions for different users for example if a user can view a page or not.
	\item \textbf{install.xml}
	This file contains all of the XML to create the database for socialwiki.
	\item \textbf{upgrade.php}
	This file tells moodle to update the database when a change has been made to the database in a previous version. For more on how upgrade.php works \url{http://docs.moodle.org/dev/Upgrade_API} 
\end{itemize}
	\subsection{folders /lang/en/}
\begin{itemize}
	\item \textbf{socialwiki.php}
	This file contains all the language strings used in the activity
	\item \textbf{parser}
	This folder contains all of the class definitions that are used to parse the wiki pages.
\end{itemize}
\subsection{pix folder}
	This folder contains most of the pictures used in the activity
	\subsection{tree\_jslib folder}
	This folder contains the JavaScript and css for the trees used to view history and search results.
	
	
	\section{Users}
	\subsection{Admin User and Teachers}
	As an admin user when you first create a socialwiki you must give the wiki a name, a description and give a title to the first page. You can also select the format of the wiki (HTML, creole or nwiki) as well as the style for the wiki (classic or modern). Once a socialwiki is created a teacher for the course must create the first page if they wish for the wiki to have a first page other wise every time a teacher views the socialwiki they will be prompted to create the first page. This however doesn't affect the general users for the course they will see the home page and will still be able to create and view pages they just won't have a link to the first page. 
	Once a Socialwiki has been created the administrative user have a few extra options that general users don't. They can go to the socialwiki administration block and change the style of the wiki or they can go to the administration tab where they have the ability to delete pages.
	\subsection{Enrolled Users}
	When an enrolled user enters a socialwiki they are taken to the home page where the can view: updated pages from the last time they logged in, pages that they created, a list of orphaned pages, a list of all pages, a list of pages recommended for them and a list of all the teacher's pages. In the navigation block there are also links that allow them to view the help page or create a new page.
	When a user views a page they can see how many likes a page has, they can like the page or follow the user who created the page. They can also click on the author and see pages the author has liked and see the authors peer scores. 
	When a user searches for pages they can view a tree that shows the history and family for the pages or they can view a list of pages or a list of pages ordered by which pages have the most likes.
	The history page allows a user to view the history of a page as well as compare two pages to see the difference between them. 
	A user can also edit a page. When a user edits a page they create a new page with any changes they made. 
	
\end{document}

